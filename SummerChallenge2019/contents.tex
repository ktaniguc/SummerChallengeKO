

\section{演習内容}

\subsection{演習1 光子を見る}
まずは干渉縞を観測する前に、そのために用いる実験装置の使い方や仕組みを学ぶために簡単に光子の測定を行う。今回の実験では光源としてLED,検出器としてMPPC,EASIROCを用いる。それぞれの詳しい説明は後で述べるためここでは省略するが、演習1でそれぞれの使い方を理解することを目的とする。
\begin{itemize}
\item LEDをパルスジェネレータを用いて光らせ、それがきちんと光っていることを目視で確認する。
\item MPPCで光子を観測する。そのためにはEASIROC,DELAY,DISCRIMINATORといったモジュールの使い方を理解して、さらにEASIROCの操作方法を理解する。
\item データを解析する。ここで簡単なROOTの使い方を理解する。
\end{itemize}

\subsection{演習2 干渉縞を観測する}
今回の実験のメインテーマとなるスリットを用いた干渉縞の観測を行う。この演習では、干渉縞を観測できるセットアップとその結果の解析がメインになると思う。
\begin{itemize}
\item レーザーポインタを用いてスリットの干渉縞を目視で確認する。実際の測定は暗箱内で行うため、常にこの干渉縞を観測しているとイメージしながら以降の測定を行ってほしい。
\item 2重スリットを用いた干渉実験。まずは干渉縞がMPPCで観測できるようなセットアップにし、稼働ステージでMPPCを移動させながら測定を行う。1回の移動で動かす距離は、理論式から明線と暗線の間隔を計算し、そこから決めるとよい。
\item ROOTを用いて測定データを解析して、干渉縞のグラフを描く。具体的な流れについては後で説明するためここでは省略する。 
\end{itemize}

\subsection{演習3 1光子の干渉縞を測定する}
演習2からの変化として光子数を減らして、1光子数での干渉縞の観測を目指す。解析手法はほとんど変わらず、光量を抑える工夫をすればよい。基本的にはLEDへの印加電圧を下げれば光量は減少するが、それでも足りなければ各自で工夫してみてください。

\clearpage
